\documentclass{jsarticle}
\usepackage{booktabs}
\usepackage{here}
\usepackage{longtable}
\begin{document}
\title{液体の比熱}
\author{有馬海人}
\maketitle

\section{実験の目的}


この実験では、試料の加熱や冷却を含む一連の操作によって液体の比熱を求める。また、熱力学第一法則やニュートンの冷却の方法を検証する。

\section{実験の原理}
ある物質の温度を$\Delta T$あげるのに必要な熱量を$\Delta Q$とすると、$\Delta Q$は$\Delta T$に比例する。
\begin{equation}
	\Delta Q = L \Delta T
\end{equation}

\begin{equation}
	L = CM
\end{equation}

と表す時、この比例定数$C$を比熱という。つまり比熱とは端子質量あたりの熱容量である。比熱の単位は$\textrm{Jkg}^{-1}\textrm{K}^{-1}$であるが、$\textrm{Jg}^{-1}\textrm{K}^{-1}$もよく使われる。なお1molあたりの熱容量をモル熱容量という。単位は$\textrm{Jmol}^{-1}\textrm{K}^{-1}$である。\\ 

\par 正確なことを言うと、一定圧力のもとで物質の温度をあげる場合と、物質の体積を一定に保って温度をあげる場合とでは熱量量、したがって比熱とは異なる。圧力を一定に保つ場合に定圧熱容量、定圧比熱といい、体積を一定に保つ場合に蹄跡熱容量、定積比熱と言う。定圧比熱$c_\textrm{p}$と定積比熱$c_textrm{v}$の比$c_\textrm{p}/c_\textrm{v}$を比熱比という。比熱比は常に1より大きい。\\

\par  液体や個体の場合には、熱膨張が期待に比べてずっと小さいので、定圧比熱と定積比熱の差はわずかである。通常は大気圧のもとでの温度変化が問題となるので、単に比熱、または熱容量といえば定圧比熱、定圧熱容量を意味する。\\

\par この実験では水とアルコールの比熱を次の2方法で測定する。

\subsection{加熱による方法}
熱量計の中に試料液体に浸された抵抗線に電流を流して温度変化を観測する。電気抵抗$R$の抵抗線に電流$i$が流れる時、時間$t$の間に発生するエネルギーは$Ri^{2}t$である。この熱量により液体と熱量計の温度が$T_0$から$T$まで上昇したとする。液体を含む熱量計の全熱量を$L$とすると、熱量計には熱量$L(T - T_0)$が加えられたことになるので、熱量計からの放熱が無視できるとすれば、次式が成り立つ

\begin{equation}
\label{3.3}
	L(TーT_0)=Ri^{2}t すなわち T(t) = T_0 + \frac{Ri^{2}}{L}t
\end{equation}

なお液体の質量を$M$、比熱を$C$、熱量計の容器の質量を$m$、比熱を$c$とすると、液体を含めた熱量計の全熱容量$L$は

\begin{equation}
	L = CM + mc
\end{equation}

\par 実際には熱量計の温度が上がってくると放熱の効果が無視できなくなるので、温度$T$を時間$t$の関数としてグラフに表すと、$T$と$t$の関係は曲線のように次第に直線からずれてくる、しかし温度が上昇し始める付近においては接線を引けば、それは\ref{3.3}式に一致するはずである。グラフから接線の傾き$\Delta T/\Delta t$を読み取れば、次の式から液体の比熱$C$を求めることができる。

\begin{equation}
\label{5}
	\frac{\Delta T}{\Delta t} = \frac{Ri^{2}}{L} = \frac{Ri^{2}}{MC + mc}
\end{equation}

\subsection{冷却による方法}
温度$T$の物質が温度$T_A$の環境の中に置かれている時、物質から環境へ移動する熱量は温度差に比例する。微小時間$\textrm{d}t$の間に環境へ移動した熱量を$\textrm{d}Q$とすると次式が成り立つ。

\begin{equation}
\label{3.6}
	\textrm{Q} = k(T- T_A)\textrm{d}t
\end{equation}

これをニュートンの冷却の法則という。比例定数$k$は放熱面の面積や形状で決まる定数である。ただし、温度差があまりに大きくなると成り立たなくなる。\\
\par 物体から熱$\textrm{d}Q$が出て行くときの物体の温度の変化$\textrm{d}T$は、物体の熱容量を$L$とすると、次式で表される。

\begin{equation}
\label{3.7}
	\textrm{d}T = -\frac{\textrm{d}Q}{L}
\end{equation}

$\textrm{d}T<)$は温度の低下を意味する。\ref{3.6}式を\ref{3.7}式に代入して

\begin{equation}
	\textrm{d}T = -\frac{k}{L}(T-T_A)\textrm{d}t
\end{equation}

すなわち物体の温度変化は次の微分方程式に従う。

\begin{equation}
	\frac{\textrm{d}T}{\textrm{d}t} = -\frac{k}{L}(T-T_A)
\end{equation}

熱量計が周囲の温度$T_A$の中に置かれているとする。はじめ$(t=0)$熱量計の温度は$T=T_m$であるとする。時間$t$の後の温度$T(t)$は上の方程式を解いて

\begin{equation}
\label{3.10}
	T(t)-T_A = (T_m-T_A)e^{-\frac{k}{L}t}
\end{equation}

となる。熱容量は$L = MC + mc$である。温度変化$T(t)$のグラフにおける曲線を冷却曲線と呼ぶ。\\
\par 水と、比熱を測定する試料液体を同じ体積だけ熱量家に入れて、それぞれについて冷却曲線を観測する。水の比熱$C_W$がわかっているとすればいかに示すいずれかの方法で試料液体の比熱を求めることができる。

\textcircled{\scriptsize 1}冷却時間の比から比熱を求める

\ref{3.10}式から温度が$T_2$から$T_1$まで冷却する時間$\Delta t$は次式で表される。

\begin{equation}
	\frac{t_\textrm L}{t_\textrm W} = -\frac{M_\textrm{L}C_\textrm L + mc}{M_\textrm W C_\textrm W +mc}
\end{equation}

$t_L$,$t_W$を求めれば、水の比熱$C_W$は既知として、試料液体の比熱$C_L$がもとまる。

\textcircled{\scriptsize 2}対数グラフの勾配の比から比熱を求める
\ref{3.10}式の両辺の対数をとると

\begin{equation}
	3.13式が入ります
\end{equation}

これより$y$と$t$の関係をグラフに書くと直線になる。2つの液体に対する勾配の比は自然対数、常用対数に関係なく

\begin{equation}
	\frac{(\Delta y/\Delta t)_L}{(\Delta y/\Delta t)_W} = -\frac{M_\textrm W C_\textrm W + mc}{M_\textrm L C_\textrm L +mc}
\end{equation}

となる。したがって対数グラフの勾配の比を求めれば、$C_W$は既知として、試料液体の比熱$C_L$を求めることができる。


\section{実験の方法}


\begin{enumerate}
	\item 恒温槽を一定温度に保つために水道の蛇口を開いた。
	\item 抵抗線をコイル状に巻いて熱量計の蓋の針金の先に半田付けをした。
	\item コイル状抵抗線の電気抵抗$R$を計測した。
	\item スイッチは切った状態で、電源,電流計,スイッチ,ヒーターコイル等を接続して電気回路を作った。
	\item 透明容器に水を張り、ヒーターコイルを浸してスイッチを入れ、電流値を調整した。
	\item 攪拌器も含めて、熱量計の質量$m$を電子天秤で計量した。
	\item 水を熱量計に入れて全体の質量を電子天秤で計量した。
	\item 水の入った熱量計を恒温槽内の断熱台の上に置いた。
	\item 最初の5分間はスイッチを入れずに温度を測定した。
	\item 撹拌を続けながらスイッチを入れた。その後、測定を継続して行なった。$0.3^\circ\textrm{C}$ごとに時間$t$を記録した。
	\item 液温が$^\circ\textrm{C}$上昇したらスイッチを切った。
	\item 恒温槽から熱量計を取り出し、水の質量をもう一度計測した。
	\item アルコールを熱量計に入れて全体の質量を電子天秤で計量した。
	\item アルコールの入った熱量計を恒温槽内の断熱台の上に置いた。
	\item 最初の5分間はスイッチを入れずに温度を測定した。
	\item 撹拌を続けながらスイッチを入れた。その後、測定を継続して行なった。$0.3^\circ\textrm{C}$ごとに時間$t$を記録した。
	\item 液温が$^\circ\textrm{C}$上昇したらスイッチを切った。
	\item 恒温槽から熱量計を取り出し、アルコールの質量をもう一度計測した。
\end{enumerate}


\section{測定結果}

\begin{table}[H]
\centering
\caption{質量,電流,抵抗の値}
\label{my-label}
\begin{tabular}{ccc}
\multicolumn{3}{c}{測定結果}             \\
         & 最初          & 最後          \\
水の質量 $/g$    & 170.76      & 170.76      \\
エタノールの質量 $/g$ & 135.25      & 134.45      \\
電流の大きさ $/A$  & \multicolumn{2}{c}{1.197} \\
抵抗の大きさ $/\Omega$  & \multicolumn{2}{c}{8.369} \\
容器の質量 $/g$  & \multicolumn{2}{c}{53.5}
\end{tabular}
\end{table}

測定結果は表   のようになった。

\if0

\begin{longtable}[h]
\centering
\caption{My caption}
\label{my-label}
\begin{longtable}{llllll}
\hline
\multicolumn{6}{|c|}{測定結果}                                                                                                                                 \\ \hline
\multicolumn{2}{|l|}{水の加熱}                         & \multicolumn{2}{l|}{エタノールの加熱}                     & \multicolumn{2}{l|}{エタノールの冷却}                     \\ \hline
\multicolumn{1}{|l|}{温度} & \multicolumn{1}{l|}{時間} & \multicolumn{1}{l|}{温度} & \multicolumn{1}{l|}{時間} & \multicolumn{1}{l|}{温度} & \multicolumn{1}{l|}{時間} \\ \hline
19.7                     & 0                       & 21.9                    & 0                       & 35                      & 0                       \\
19.8                     & 60                      & 21.8                    & 30                      & 34.7                    & 51                      \\
19.8                     & 120                     & 21.8                    & 60                      & 34.4                    & 101                     \\
20                       & 180                     & 21.8                    & 90                      & 34.1                    & 155                     \\
20.2                     & 240                     & 21.7                    & 120                     & 33.8                    & 210                     \\
20.3                     & 300                     & 21.7                    & 150                     & 33.5                    & 270                     \\
20.6                     & 360                     & 21.7                    & 180                     & 33.2                    & 332                     \\
20.9                     & 380                     & 21.7                    & 210                     & 32.9                    & 402                     \\
21.2                     & 401                     & 21.6                    & 240                     & 32.6                    & 468                     \\
21.5                     & 420                     & 21.6                    & 270                     & 32.3                    & 546                     \\
21.8                     & 441                     & 21.6                    & 300                     & 32                      & 625                     \\
22.1                     & 460                     & 21.6                    & 300                     & 31.7                    & 706                     \\
22.4                     & 481                     & 21.9                    & 312                     & 31.4                    & 789                     \\
22.7                     & 504                     & 22.2                    & 321                     & 31.1                    & 880                     \\
23                       & 521                     & 22.5                    & 331                     & 30.8                    & 977                     \\
23.3                     & 539                     & 22.8                    & 340                     & 30.5                    & 1079                    \\
23.6                     & 559                     & 23.1                    & 350                     & 30.2                    & 1152                    \\
23.9                     & 578                     & 23.4                    & 359                     & 29.9                    & 1220                    \\
24.2                     & 598                     & 23.7                    & 370                     & 無し                      & 無し                      \\
24.5                     & 617                     & 24                      & 379                     & 無し                      & 無し                      \\
24.8                     & 638                     & 24.3                    & 388                     & 無し                      & 無し                      \\
25.1                     & 654                     & 24.6                    & 398                     & 無し                      & 無し                      \\
25.4                     & 673                     & 24.9                    & 407                     & 無し                      & 無し                      \\
25.7                     & 691                     & 25.2                    & 416                     & 無し                      & 無し                      \\
26                       & 708                     & 25.5                    & 426                     & 無し                      & 無し                      \\
26.3                     & 728                     & 25.8                    & 434                     & 無し                      & 無し                      \\
26.6                     & 745                     & 26.1                    & 444                     & 無し                      & 無し                      \\
26.9                     & 762                     & 26.4                    & 453                     & 無し                      & 無し                      \\
27.2                     & 778                     & 26.7                    & 463                     & 無し                      & 無し                      \\
27.5                     & 798                     & 27                      & 472                     & 無し                      & 無し                      \\
27.8                     & 818                     & 27.3                    & 479                     & 無し                      & 無し                      \\
28.1                     & 833                     & 27.6                    & 490                     & 無し                      & 無し                      \\
28.4                     & 854                     & 27.9                    & 499                     & 無し                      & 無し                      \\
28.7                     & 870                     & 28.2                    & 507                     & 無し                      & 無し                      \\
29                       & 886                     & 28.5                    & 517                     & 無し                      & 無し                      \\
29.3                     & 902                     & 28.8                    & 523                     & 無し                      & 無し                      \\
29.6                     & 921                     & 29.1                    & 533                     & 無し                      & 無し                      \\
29.9                     & 937                     & 29.4                    & 542                     & 無し                      & 無し                      \\
30.2                     & 955                     & 29.7                    & 551                     & 無し                      & 無し                      \\
30.5                     & 971                     & 30                      & 558                     & 無し                      & 無し                      \\
30.8                     & 995                     & 30.3                    & 568                     & 無し                      & 無し                      \\
31.1                     & 1025                    & 30.6                    & 578                     & 無し                      & 無し                      \\
31.4                     & 1050                    & 30.9                    & 591                     & 無し                      & 無し                      \\
31.7                     & 1077                    & 31.2                    & 604                     & 無し                      & 無し                      \\
32                       & 1105                    & 31.5                    & 618                     & 無し                      & 無し                      \\
32.3                     & 1132                    & 31.8                    & 632                     & 無し                      & 無し                      \\
32.6                     & 1158                    & 32.1                    & 646                     & 無し                      & 無し                      \\
32.9                     & 1184                    & 32.4                    & 660                     & 無し                      & 無し                      \\
33.2                     & 1210                    & 32.7                    & 674                     & 無し                      & 無し                      \\
33.5                     & 1238                    & 33                      & 687                     & 無し                      & 無し                      \\
33.8                     & 1263                    & 33.3                    & 700                     & 無し                      & 無し                      \\
34.1                     & 1288                    & 33.6                    & 714                     & 無し                      & 無し                      \\
34.4                     & 1314                    & 33.9                    & 728                     & 無し                      & 無し                      \\
34.7                     & 1341                    & 34.2                    & 741                     & 無し                      & 無し                      \\
無し                       & 無し                      & 34.5                    & 756                     & 無し                      & 無し                      \\
無し                       & 無し                      & 34.8                    & 767                     & 無し                      & 無し                      \\
無し                       & 無し                      & 35.1                    & 778                     & 無し                      & 無し                      \\
無し                       & 無し                      & 35.4                    & 800                     & 無し                      & 無し                     
\end{longtable}
\end{longtable}
\fi


\section{課題}

比熱を求めるために\ref{5}式より、比熱$C$は次式のようになる。

\begin{equation}
	C = \frac{1}{M}(\frac{Ri^2\Delta t}{\Delta T}-mc)
\end{equation}

これを合成標準不確かさの式に当てはめると次のようになる。

\begin{equation}
	\Delta C = \sqrt{(\frac{\delta C}{\delta M})(\Delta M)^2 + (\frac{\delta C}{\delta R})(\Delta R)^2 + (\frac{\delta C}{\delta i})(\Delta i)^2 + (\frac{\delta C}{\delta \Delta t})(\Delta(\Delta t))^2 + (\frac{\delta C}{\delta m})(\Delta m)^2}
\end{equation}

また、$\frac{\delta C}{\delta M} = -\frac{1}{M}(\frac{\Delta t}{\Delta T}Ri^2-mc)$,$\frac{\delta C}{\delta i} = \frac{1}{M}\frac{2\Delta tRi}{\Delta}$,$\frac{\delta C}{\delta i} = -\frac{2\Delta tRi}{M\Delta t}$,$\frac{\delta C}{\delta(\Delta t)} = \frac{Ri^2}{M\Delta T}$,$\frac{\delta C}{\delta(\Delta T)}=-\frac{Ri^2\Delta t}{M(\Delta T)^2}$,$\frac{\delta c}{\delta m}=-\frac{c}{M}$である。さらに、$i = 1.197$,$R = 8.369$,水の質量$M_W = 170.76$,エタノールの質量$M_L  = 135.25$,容器の質量$m = 53.5$,容器の比熱$c = 0.385$,$\Delta M = 0.01 , \Delta R = 0.001 , \Delta i = 0.001 , \Delta(\Delta t) = 0.1 , \Delta(\Delta T) = 0.1 , \Delta m = 0.01$であるとして、比熱の合成不確さを計算する。ここで水の比熱の合成不確かさを$\Delta C_W$エタノールの比熱の合成不確かさを$\Delta C_L$とすると$\Delta C_W = 0.08$,$\Delta C_L = 0.06$となった。

次に実験によって得られた値から比熱を導出する。
まず、水の温度上昇の曲線の接線の傾きは$\frac{\Delta T_W}{\Delta t_w} = \frac{33.75-20.63}{1133.3-366.6} = 0.01357$,エタノールの温度上昇の曲線の接線の傾きは$\frac{\Delta T_L}{\Delta t_L} = \frac{30.67-22.541}{572.0-327.3} = 0.0332$である。そして、$M_WC_W + mc = \frac{Ri^2\Delta t_W}{\Delta T_W} = \frac{8.369*1.197*1.197}{0.01357} = 884.30,M_LC_L + mc = \frac{Ri^2\Delta t_L}{\Delta T_L} = \frac{8.369*1.197*1.197}{0.0332} = 361.44$

ここで$C = \frac{1}{M}(\frac{Ri^2\Delta t}{\Delta T}-mc)$なのでこれにそれぞれの値を代入する。

$C_W = \frac{1}{M_W}(\frac{Ri^2\Delta t_W}{\Delta T_W}-mc) = \frac{1}{170.76}(884.30-0.385*53.5) = 5.058$

$C_L = \frac{1}{M_L}(\frac{Ri^2\Delta t_L}{\Delta T_L}-mc) = \frac{1}{135.25}(361.44-0.385*53.5) = 2.520$


以上より合成不確かさを踏まえたそれぞれの試料の比熱は、水の比熱$C_W = 5.058 \pm 0.08$,エタノールの比熱$C_L = 2.520 \pm 0.06$である。



\section{考察}

\subsection{(a)}
私の冷却時に水とアルコールの体積が同じでないといけない理由は以下のものあると考察する。
1つ目の理由は2つの試料液体の体積が異なる場合に熱容量などの別の要因を考量する必要があるからだと考察する。これらが異なる場合はそれらの値により良い正確さが求められてしまうため、実験でやった内容のこと以外にも値の誤差などを考慮する際に不確かさなどの計算が複雑になってしまうからであると考察する。
\subsection{(b)}
水、エタノールそのものの体積の変化を$\Delta m_W = 0.29$,$\Delta m_L = 0.2$とすると、$\Delta C_W’ = \frac{\sqrt{M_W}}{\sqrt{M_W + m_W}}C_W = 5.049$,$\Delta C_L’ = \frac{\sqrt{M_L}}{\sqrt{M_L + m_L}}C_L = 2.516$になると考察する。なぜなら、不確かさの式に試料の大きさは、試料の大きさを$W$とすると、$\frac{1}{\sqrt{W}}$分しか関与していない。したがって、不確かさの式の中にある$\frac{1}{\sqrt{W}}$を打ち消すために$\sqrt{W}$を掛け、それを埋めるために$sqrt{W+w}$で割れば問題ないと考察したからである。
\subsection{(c)}
今回は容器から恒温槽に出て行く熱について考えていないと考察する。しかし、恒温槽と液体試料の温度差も大きくはなく、温度差に与えた影響を$0.1$Kであったとしても不確かさの式を通して大きくとも$\sqrt{0.001*0.1^2} = 0.1^{\frac{5}{2}} = 0.003$程度であると考察する。
\subsection{(d)}
今回の実験では不確かさの範囲内でも出した値が文献値とかぶることがなかったのでこの理由を考察する。水の場合とエタノールの場合では水の方が実験前と実験後での質量の減りが大きかった。また、水の場合とエタノールの場合では水の方が文献値と実験値の違いが大きかった。このことから、質量の変化量と、文献値と実験値の違いの大きさには相関関係があると考察する。したがって、この文献値と実験値
の違いの原因は実験の前後での液体試料の質量の差にあると考察する。そして、液体試料の質量の差が生じる原因としては実験中はずっと行ってなくてはならない攪拌操作であると考察する。よってこのような差をなくすには攪拌操作をより慎重に行う必要があると考察する。

\begin{thebibliography}{99}
\bibitem{UEC} 共通教育部自然科学部会(物理)、『基礎科学実験A(物理学実験)平成29年(2017年)版』
\end{thebibliography}

\newpage

\subsection{追加の考察}
\subsection{6.4(d)}

\begin{table}[H]
	\centering
	\caption{テキスト55ページにある比熱の文献値}
	\label{my-label}
	\begin{tabular}{llll}
	\hline
	比熱 /Jg\textasciicircum -1K\textasciicircum -1 & 0 $^\circ$C    & 15  $^\circ$C  & 25  $^\circ$C  \\ \hline
	銅                                             & 0.38  & 0.383 & 0.385 \\
	水                                             & 4.218 & 4.186 & 4.178 \\
	エタノール                                         & 2.31  & 2.39  & 2.57 
	\end{tabular}
\end{table}

また、以上の文献値を用いて水とエタノールの比熱それぞれに対して相対誤差を求めた。水の比熱の相対誤差は0.21で、エターノルの比熱の相対誤差は0.019となった。この結果において相対誤差の小さいエタノールは合計3回目の比熱の計測の実験であったからだと考察する。また、エタノールは最初の実験の際に失敗したため、2回目の今回の実験ではより慎重になり、実験にもなれたため、この結果が出たのであると考察する。

\subsection{エタノールの冷却曲線について}
\ref{3.10}より、$t\to \infty$をすると$T(t) = T_A$となる。これは十分な時間が経つと、液体試料の温度を表す関数は室温($T_A$)に収束するということである。図3の冷却曲線も室温(21.5度)に近くにつれて値の減少速度が緩やかになっているのがわかる。したがってこの冷却曲線はニュートンの冷却の法則の範囲に収まる曲線であると考察する。

\end{document}