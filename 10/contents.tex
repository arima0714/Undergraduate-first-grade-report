\documentclass{jsarticle}
\usepackage{booktabs}
\usepackage{here}
\usepackage{url}
\begin{document}
\title{光速度}
\author{有馬海人}
\maketitle

\section{実験の目的}

光速度の直接測定と同軸ケーブルを信号が伝わる速度の測定を行う。

\section{実験の原理}

    この実験で使う半導体レーザーは波長685nmの赤色光を出す。このレーザー変調入力端子に約1~2Vの電圧を加えるとレーザー光はでない。そこでぱるす発生器を使うとレーザーからの出力は幅約10nsの光パルスとなる。\\
    \par このような短い光パルスを観測するためには応答時間が1ns以下の光検出器を使用しなければならない。ここではpinフォトダイオードを用いる。受光面に当たった光パルスと同じ波形の電気信号を出力する。\\
    \par 測定は、半導体レーザーからの、繰り返し周波数やく15MHzの光パルス列を数m離れたプリズム反射機で反射させて光検出器に導く。光検出器の出力信号をオシロスコープに導き、2つの光パルス時間間隔$T$を測定する。一方、プリズム反射機または半透明鏡で反射して光検出器に入る光路の差$L$を測定する。光速度$c$は次の式で計算される。
\begin{equation}
\label{lighrningspeed}
    c = \frac{L}{T}
\end{equation}

\section{実験の方法}

\subsection{光速度の測定}

\subsubsection{$T$の観測}
\begin{enumerate}
    \item ケーブルが指示通りに繋がっているか確認した。
    \item 装置の電源を入れて半導体レーザーから連続光を出した。
    \item レーザー光をプリズム反射機の左側のプリズムに当てた。
    \item 反射光を光検出器の受光面の近くに戻るようプリズム反射機の上下の傾きを調整した。
    \item 反射レーザー光が光検出器の受光面に正しく当たるようにし、オシロスコープの画面を見ながら光検出器の位置を調整した。
    \item 半導体レーザーからおよそ60cmほど離して、レーザー光線に半透明鏡を挿入し、レーザー光線が半透明鏡の中央を通るように位置を調整した。
    \item 半透明鏡の反射光が光検出器の受光面に当たるように、半透明鏡の向きを調整した。
    \item 半透明鏡と光検出器の中間付近に焦点距離30cmのレンズを挿入し、半透明鏡の反射光が光検出器の受光面に当たるように、レンズの位置を調整した。
    \item オシロスコープで2つのパルスが観測できるようになったら、その時間間隔$T$を測定した。
\end{enumerate}

\subsubsection{光路の長さの測定}
\begin{enumerate}
    \item 半透明鏡からプリズム反射機のプリズム面までの距離$d_1$,光検出器からプリズム面までの距離$d_2$,半透明鏡から光検出器までの距離$d_3$を巻尺を使って測定した。
    \item プリズム反射機の第一プリズムに入ってから二番目のプリズムから出てくるまでの幾何学的な光路の長さは、光がどこに入ろうとも$l_0$であるのに注意しながら、$l_0$を測定した。
\end{enumerate}

\subsubsection{水中の場合の測定}
\begin{enumerate}
    \item 光路$d_2$に水の入ったガラス管を挿入した。
    \item 先ほどと同様に時間間隔$T$
\end{enumerate}

\subsection{同軸ケーブルを伝わる信号速度の測定}
\begin{enumerate}
    \item オシロスコープの電源を入れた。
    \item パルス発生器の電源を入れた。
    \item 同軸ケーブルの終端に付いているインピーダンス整合機をはずした。
    \item 直接来たパルスと終端の反射によって生じたパルスとの間の時間間隔$T$を測定した。
    \item 同軸ケーブルを取り外してその長さ$L$を測定した。
\end{enumerate}

\section{測定結果}

半透明鏡からプリズム反射機のプリズム面までの距離を$d_1$、光検出器からプリズム面までの距離を$d_2$、半透明鏡から光検出器までの距離を$d_3$とする。また、光路の差$L$は次の式から出した。ガラスの屈折率は1.5としている。

\begin{equation}
    L = d_1 + d_2 - d_3 + 1.5(l_0 - l_1) + l_1
\end{equation}

\begin{table}[H]
\centering
\caption{光速度の測定における測定結果}
\label{one}
\begin{tabular}{|l|l|}
\hline
$T $ /s & 17.9    \\ \hline
$d_1$ /cm & 249.6   \\ \hline
$d_2$ /cm & 309.8   \\ \hline
$d_3$ /cm & 60.7    \\ \hline
$l_0$ /cm & 17.81   \\ \hline
$l_1$ /cm & 12.8    \\ \hline
$L $ /cm & 519.015 \\ \hline
\end{tabular}
\end{table}

\begin{table}[H]
\centering
\caption{水中の場合における測定結果}
\label{two}
\begin{tabular}{|l|l|}
\hline
$T   $/s & 19.3    \\ \hline
$d_1$/cm & 249.6   \\ \hline
$d_2$/cm & 309.8   \\ \hline
$d_3$/cm & 60.7    \\ \hline
$l_0$/cm & 17.81   \\ \hline
$l_1$/cm & 12.8    \\ \hline
$L   $/cm & 519.015 \\ \hline
\end{tabular}
\end{table}

\begin{table}[H]
\centering
\caption{同軸ケーブルでの信号速度における測定結果}
\label{three}
\begin{tabular}{|l|l|}
\hline
$T$/s  & 10.9  \\ \hline
$L$/cm & 103.8 \\ \hline
\end{tabular}
\end{table}


\section{課題}

\subsection{光速度とその不確かさ}
式\ref{lighrningspeed}より、表\ref{one}の値を代入すると、光速度$c = 2.900 \times 10^8$m/sとなった。\\
\par 次に光速度の不確かさを求めていく。不確かさを出すのには次の式を使用した。また、この時の$\Delta L = 0.1$,$\Delta T = 0.5$である。

\begin{equation}
    \Delta c = c\sqrt{(\frac{\Delta L}{L})^2 + (\frac{\Delta T}{T})^2}
\end{equation}

以上より、$\Delta T = 0.08 \times 10^8$m/sとなった。

従ってこの実験から得られた光速度$c = 2.900 \pm 0.08 \times 10^8$m/sである。




\subsection{水の屈折率とその不確かさ}
水の屈折率を$n$とすると、水の屈折率は次の式で計算される。
\begin{equation}
    n = 1 + \frac{ct}{l}
\end{equation}
上式と\ref{two}の値より、水の屈折率$n = 1.68$となった。\\
\par 次に水の屈折率の不確かさを求めていく。不確かさを出すのには次の式を利用した。また、この時の$\Delta t = 0.1$,$\Delta l = 0.1$である。

\begin{equation}
    \Delta n = (n-1)\sqrt{(\frac{\Delta t}{t})^2 + (\frac{\Delta c}{c})^2 + (\frac{\Delta l}{l})^2}
\end{equation}

上式に表\ref{two}の値を代入することにより、$\Delta n = 0.05$となった。

従ってこの実験から得られた水の屈折率$n = 1.68 \pm 0.05$である。




\subsection{同軸ケーブルの信号速度とその不確かさ}
時間間隔を$T$同軸ケーブルの長さを$L$とすると信号の速度$v$は次式より表すことができる。

\begin{equation}
    v = \frac{2L}{T}
\end{equation}

上式に\ref{two}の値を代入した結果、信号の速度$v = 1.90 \times 10^8$m/sとなった。\\
\par 次に信号の速度の不確かさを求めていく。不確かさを出すのには次の式を用いた。このとき、$\Delta T = 0.5$,$\Delta L = 0.1$であるとした。

\begin{equation}
    \Delta v = v\sqrt{(\frac{\Delta L}{L})^2 + (\frac{\Delta T}{T})^2}
\end{equation}
上式に表\ref{two}の値を代入することにより、$\Delta v = 0.09 \times 10^8$m/sとなった。

したがって、今回の実験から得られた同軸ケーブル内の信号速度は$v = 1.90 \pm 0.09 \times 10^8$m/sである。



\section{考察}

\subsection{課題4について}
私は測定した光速度について、’空気中’であって’真空中’であることを問題にする必要はないと考察する。その理由は、実験で得られた値のそれぞれが屈折率の有効数字に及ばないからである。従って、今回の実験において得られた光速度は’空気中’であるか’真空中’であるかを問題にする必要はないと考察する。

\begin{table}[H]
\centering
\caption{実験結果}
\label{my-label}
\begin{tabular}{|c|c|c|c|}
\hline
             & 実験値                & 理論値         & 相対誤差  \\ \hline
光速度  m/s        & $2.900 \pm 0.08 \times 10^8$ & $2.9979 \times 10^8$ & 0.0059 \\\hline
水の屈折率        & $1.68 \pm 0.05$       & 1.3330      & 0.26 \\ \hline
同軸ケーブル内の信号速度 m/s & 1.90 pm 0.09 1068  &             &      \\ \hline
\end{tabular}
\end{table}

全体の実験結果は上記のような表となった。\\
\par まずは光速度について考察していく。この光速度の相対誤差から、この実験における光速度を求めるという目標は達成できたのではないかと考察する。
\par 次には水の屈折率について考察していく。この水の屈折率の相対誤差から、三割を下回っているので、文献値に、今回の光速度の結果ほどではないが非常に良い値が出ているのではないかと考察する。そして、このような値の誤差が出てしまった原因を考察する。1つめには予習をしっかりしていなかったことであると考察する。なぜなら、webclassにおいて水の入った管の説明がなされていたのにもかかわらず、それがあったのを失念していたために管の調整に時間がかかってしまったからであると考察する。2つ目には水の入った管の調整が非常にシビアであることである。光路が当然ながら、直線であるためにその光線の中に物体を入れることが自分たちにとってはとても慎重を要するものであったからであると考察する。これをなくすためのアイデアとして次のものがある。少しながら線香などで煙を発生させることで光路をはっきりとさせることであると考察する。こうすることで普段では見ることのできない、光路を出現させることで水の入った管をよりたやすく、光路上に挿入することができるのではないかと考察するからである。



\begin{thebibliography}{99}
    \bibitem{UEC} 共通教育部自然科学部会(物理)、『基礎科学実験A(物理学実験)平成29年(2017年)版』
\end{thebibliography}

\end{document}