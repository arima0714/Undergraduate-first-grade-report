\documentclass{jsarticle}
\usepackage{booktabs}
\usepackage{here}
\usepackage{url}
\begin{document}
\title{光速度}
\author{有馬海人}
\maketitle

\section{実験の目的}

光速度の直接測定と同軸ケーブルを信号が伝わる速度の測定を行う。

\section{実験の原理}

    この実験で使う半導体レーザーは波長685nmの赤色光を出す。このレーザー変調入力端子に約1~2Vの電圧を加えるとレーザー光はでない。そこでぱるす発生器を使うとレーザーからの出力は幅約10nsの光パルスとなる。\\
    \par このような短い光パルスを観測するためには応答時間が1ns以下の光検出器を使用しなければならない。ここではpinフォトダイオードを用いる。受光面に当たった光パルスと同じ波形の電気信号を出力する。\\
    \par 測定は、半導体レーザーからの、繰り返し周波数やく15MHzの光パルス列を数m離れたプリズム反射機で反射させて光検出器に導く。光検出器の出力信号をオシロスコープに導き、2つの光パルス時間間隔$T$を測定する。一方、プリズム反射機または半透明鏡で反射して光検出器に入る光路の差$L$を測定する。光速度$c$は次の式で計算される。
\begin{equation}
    c = \frac{L}{T}
\end{equation}

\section{実験の方法}

\subsubsection{$T$の観測}
\begin{enumerate}
    \item ケーブルが指示通りに繋がっているか確認した。
    \item 装置の電源を入れて半導体レーザーから連続光を出した。
    \item レーザー光をプリズム反射機の左側のプリズムに当てた。
    \item 反射光を光検出器の受光面の近くに戻るようプリズム反射機の上下の傾きを調整した。
    \item 反射レーザー光が光検出器の受光面に正しく当たるようにし、オシロスコープの画面を見ながら光検出器の位置を調整した。
    \item 半導体レーザーからおよそ60cmほど離して、レーザー光線に半透明鏡を挿入し、レーザー光線が半透明鏡の中央を通るように位置を調整した。
    \item 半透明鏡の反射光が光検出器の受光面に当たるように、半透明鏡の向きを調整した。
    \item 半透明鏡と光検出器の中間付近に焦点距離30cmのレンズを挿入し、半透明鏡の反射光が光検出器の受光面に当たるように、レンズの位置を調整した。
    \item オシロスコープで2つのパルスが観測できるようになったら、その時間間隔$T$を測定した。
\end{enumerate}

\subsubsection{光路の長さの測定}
\begin{enumerate}
    \item 半透明鏡からプリズム反射機のプリズム面までの距離$d_1$,光検出器からプリズム面までの距離$d_2$,半透明鏡から光検出器までの距離$d_3$を巻尺を使って測定した。
    \item プリズム反射機の第一プリズムに入ってから二番目のプリズムから出てくるまでの幾何学的な光路の長さは、光がどこに入ろうとも$l_0$であるのに注意しながら、$l_0$を測定した。
    \item 
\end{enumerate}

\section{測定結果}

測定結果

\section{課題}

課題

\section{考察}

考察






\begin{thebibliography}{99}
    \bibitem{UEC} 共通教育部自然科学部会(物理)、『基礎科学実験A(物理学実験)平成29年(2017年)版』
\end{thebibliography}

\end{document}