\documentclass{jsarticle}
\usepackage{booktabs}
\usepackage{here}
\usepackage{url}
\begin{document}
\title{実験名}
\author{有馬海人}
\maketitle

\section{実験の目的}

光電管を用いてプランク定数を測定する。また、光電効果の諸特性について得られた測定値を元に検証を行う。

\section{実験の原理}

金属表面に特定の振動数以上の光が当たると電子が飛び出す現象は光電効果として知られている。光電子の運動エネルギー$K$と光の振動数$\nu$,金属表面から光電子が飛び出してくるのに必要なエネルギーを$W$とすると、これらの量には次式の関係が成り立つ。

\begin{equation}
    K = h\ne -W
\end{equation}

ここで、$h$はプランク定数である。光電効果は次のような性質を持つことがわかっている。

\begin{enumerate}
    \item 金属の種類によって決まる限界振動数より小さい振動数の光では、光の強度を強くしても光電子は観測されない。逆に、光の振動数が限界振動数より大きければ、光の強度がどんなに小さくても光電子は観測される。
    \item 光電子の運動エネルギーは光の強度によらず、光の振動数のみに依存する。
    \item 光の振動数を一定にし、その強度を強くしていくと光電子の数が増加するが、光電子の運動エネルギーは変化しない。
\end{enumerate}

光電管は光電効果を利用して光エネルギーを電気エネルギーに変換する電子管で、ほぼ真空のガラス管内に入れた陽極と陰極からなる。陰極に限界振動数より高い振動数の光を当てると、光電子が飛び出し陽極へ到達し、電流として取り出すことができる。この電流を光電流と呼ぶ。陰極には仕事関数が小さいアルカリ金属が用いられる。\\
\par 以上から、光電管を用いて光電子の運動エネルギーを測定すれば、プランク定数を求めることができる。また、同様の測定を条件を変えて光電流を測ることで、上記の光電効果の諸性質を定量的に検証することができる。

\begin{equation}
    実験の原理
\end{equation}

\begin{equation}
    実験の原理
\end{equation}

\section{実験の方法}

実験の方法

\section{測定結果}

測定結果

\section{課題}

課題

\section{考察}

考察






\begin{thebibliography}{99}
    \bibitem{UEC} 共通教育部自然科学部会(物理)、『基礎科学実験A(物理学実験)平成29年(2017年)版』
\end{thebibliography}

\end{document}