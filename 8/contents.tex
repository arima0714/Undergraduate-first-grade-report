\documentclass{jsarticle}
\usepackage{booktabs}
\usepackage{here}
\usepackage{url}
\begin{document}
\title{光電効果}
\author{有馬海人}
\maketitle

\section{実験の目的}

光電管を用いてプランク定数を測定する。また、光電効果の諸特性について得られた測定値を元に検証を行う。

\section{実験の原理}

金属表面に特定の振動数以上の光が当たると電子が飛び出す現象は光電効果として知られている。光電子の運動エネルギー$K$と光の振動数$\nu$,金属表面から光電子が飛び出してくるのに必要なエネルギーを$W$とすると、これらの量には次式の関係が成り立つ。

\begin{equation}
    K = h\nu -W
\end{equation}

ここで、$h$はプランク定数である。光電効果は次のような性質を持つことがわかっている。

\begin{enumerate}
    \item 金属の種類によって決まる限界振動数より小さい振動数の光では、光の強度を強くしても光電子は観測されない。逆に、光の振動数が限界振動数より大きければ、光の強度がどんなに小さくても光電子は観測される。
    \item 光電子の運動エネルギーは光の強度によらず、光の振動数のみに依存する。
    \item 光の振動数を一定にし、その強度を強くしていくと光電子の数が増加するが、光電子の運動エネルギーは変化しない。
\end{enumerate}

光電管は光電効果を利用して光エネルギーを電気エネルギーに変換する電子管で、ほぼ真空のガラス管内に入れた陽極と陰極からなる。陰極に限界振動数より高い振動数の光を当てると、光電子が飛び出し陽極へ到達し、電流として取り出すことができる。この電流を光電流と呼ぶ。陰極には仕事関数が小さいアルカリ金属が用いられる。\\
\par 以上から、光電管を用いて光電子の運動エネルギーを測定すれば、プランク定数を求めることができる。また、同様の測定を条件を変えて光電流を測ることで、上記の光電効果の諸性質を定量的に検証することができる。\\
\par 光電管には、光電面から飛び出した光電子を陽極から追い返す向き、あるいは加速する向きに電圧を変化させながら印可できるよう工夫されている。光電管に対する印可電圧と陽極より取り出した光電流はそれぞれ電圧計と電流計によって測定できる。\\
\par 適当な光源を用意して光電管に当てて、光電面から飛び出した光電子に対して逆方向電圧$V$をかけると、光電子はクーロン斥力のために陽極に到達しにくくなる。印可電圧をあげると、ついには光電子は全て電圧によって追い返され陽極に到達できなくなる。すなわち、光電流はゼロとなる。この時の電圧を$V_0$とすると、電圧によって光電子が失うエネルギーと光電子が陰極を飛び出した瞬間に持っていた運動エネルギーが等しいので、$eV_0 = K$となる。従って$V_0$を測定することで、光電子の最大エネルギーの大きさを知ることができる。なお、電子など荷電粒子のエネルギーの大きさは電子ボルトがよく用いられる。1電子ボルトは

\begin{equation}
    1\textrm{eV} = 1.6022 \times 10^{-19} \textrm{J}
\end{equation}

である。これは1つの電子が1Vの電圧によって加速されるときに得る運動エネルギーの大きさに等しい。

\section{実験の方法}

\subsection{準備}
\begin{enumerate}
\item NULLキーをoff,ランプを点灯し,光電管に2Vを印加した。
\item 赤色フィルタを挿入し電流値を10回測定した。
\item 橙,黄,青色フィルタについて同じ測定を行った。
\item 全測定値中で一番大きな電流値が出現するフィルタを挿入した。
\item 表示される測定値が測定された最大値と同程度となったときNULLキーを押した。
\item 数回値を観測して平均的に負値となることを確認した。
\end{enumerate}
\subsection{測定1}
\begin{enumerate}
\item 絞板は挿入せず,赤色フィルタのみ挿入した。
\item 電圧を0Vにして光電流値を測定した。
\item 電圧を0.05V毎に変え,その都度光電流を測定した。
\item 光電流値の平均値の前の値との差が5点以上変化しなくなるまで測定を繰り返した。
\item 手順(2)以降の手順で測定を繰り返した。
\end{enumerate}
\subsection{測定2}
\begin{enumerate}
\item 7mmの絞板を挿入した。
\item 黄,青フィルタについて測定1の(2)~(4)を行った。
\end{enumerate}
\subsection{測定3}
\begin{enumerate}
\item 印加電圧を0Vとし,色フィルタをはずした。
\item 20mmの絞板の光電流値を測定した。
\item 全ての絞板で同様に光電流値を記録した。
\end{enumerate}
\section{測定結果}

測定結果

\section{課題}

課題

\section{考察}

考察






\begin{thebibliography}{99}
    \bibitem{UEC} 共通教育部自然科学部会(物理)、『基礎科学実験A(物理学実験)平成29年(2017年)版』
\end{thebibliography}

\end{document}