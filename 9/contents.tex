\documentclass{jsarticle}
\usepackage{booktabs}
\usepackage{here}
\usepackage{url}
\usepackage{lscape}
\begin{document}
\title{光のスペクトル}
\author{有馬海人}
\maketitle

\section{実験の目的}

光源にNaランプやHgランプなどを用い、回折格子分光計を使って、それぞれの原子に特有なスペクトル線を観測し、その波長を用いる。

\section{実験の原理}

\subsection{スペクトル線について}

量子理論によれば、水素原子のエネルギーは次のとびとびの値しかとることができない。

\begin{equation}
    E_n = -\frac{hcR}{n^2} (n = 1,2,3,\cdots)
\end{equation}

ここで$hcR$は定数である。$h$はプランク定数,$c$は真空中の光速度で、$R$はリュードベリ定数と呼ばれる。整数$n$で決められる原子状態及びエネルギーの値をエネルギー準位という。電気放電などによって高いエネルギー準位$n_1$に上げられた原子がより低いエネルギー準位$n_2$へ状態を変えるときに光が放出され、その周波数$\nu$と波長$\lambda$は次の式から決定される。

\begin{equation}
    h\nu = E_{n_1} - E_{n_2} 及び \frac{1}{\lambda} = \frac{\nu}{c} = \frac{1}{hc} (E_{n_1} - E_{n_2})
\end{equation}

水素原子の可視域のスペクトル線は$n > 2$の順位から$n = 2$の順位へ遷移するときに放出され、

\begin{equation}
    \frac{1}{\lambda} = \frac{1}{hc}(E_{n_1} - E_{n_2}) = R(\frac{1}{2^n}{2^n}) (n = 3,4,5,\cdots)
\end{equation}

から計算される。

\subsection{回折格子の原理}

スリットを狭い間隔で平行に並べたものを回折格子という。特定の波長の平行光線を回折格子の面に垂直に当てると、回折格子を通った光はいくつかの方向に現れる。入射方向に出てくる光線を0次,それより角度のます方向に順次1次,2次,$\cdots$の回折光という。入射光の方向と回折光の方向のなす角を回折角という。m次回折角$\theta_m$は次の式で与えられる。

\begin{equation}
    d\sin{\theta_m} = m\lambda または \sin{\theta_m} = m\lambda N (m = \pm 1,\pm 2, \cdots)
\end{equation}

ここで$\lambda$は光の波長,$d$は回折格子の格子の間隔,$N = 1/d$はdの逆数で単位長さあたりの格子の数である。回折格子の特性を表す時は$d$よりも$N$を用いるのが普通である。上式は隣接するスリットを通った光が強め合う条件である。色々な波長を含む光を回折格子に当てると回折光は分かれ、入射した光のスペクトルが観測される。次数が同じ回折行を調べると波長の長い赤いスペクトル線の方が波長の短い青いスペクトル線よりも回折角が大きいことに注意する。



\section{実験の方法}

\subsubsection{視差をなくすための調整}
\begin{enumerate}
    \item 十時線がはっきり見えるように接眼レンズを前後に動かし、ピントを合わせた。
    \item スリットの輪郭がはっきり見えるように伸縮ネジ$R_1$を動かした。
\end{enumerate}

\subsubsection{回折格子の置き方}
\begin{enumerate}
    \item 目盛盤を固定し望遠鏡を可動にした。
    \item 回折格子を回転台に載せ、情報から見て回折格子の格子面をコリメータの軸と可能な限り垂直になるように回転台を回した。
\end{enumerate}

\subsubsection{回折角の測定}
\begin{enumerate}
    \item スペクトル線を発見したら、スペクトル線を正しく十字線の交点に合わせその時の目盛を読み取った。
    \item 種々のスペクトル線の一次、二次の回折角$\theta_1,\theta_2$を測定していった。
\end{enumerate}



\section{測定結果}


\begin{table}[H]
\centering
\caption{$\textrm{N}_\textrm{a}$}
\label{my-label}
\begin{tabular}{|c|c|c|}
\hline
     & θL      & θR      \\ \hline
橙 一次 & 345°00' & 303°30' \\ \hline
橙 二次 & 369°46' & 279°37' \\ \hline
橙 一次 & 344°57' & 303°33' \\ \hline
橙 二次 & 369°40' & 279°42' \\ \hline
\end{tabular}
\end{table}

\begin{table}[H]
\centering
\caption{H}
\label{my-label}
\begin{tabular}{|c|c|c|}
\hline
     & θL      & θR      \\ \hline
赤 一次 & 346°01' & 301°03' \\ \hline
赤 二次 & 376°57' & 272°57' \\ \hline
緑 一次 & 341°12' & 307°17' \\ \hline
緑 二次 & 360°10' & 288°46' \\ \hline
青 一次 & 339°19' & 309°06' \\ \hline
青 二次 & 355°46' & 293°00' \\ \hline
\end{tabular}
\end{table}

\begin{table}[H]
\centering
\caption{$\textrm{H}_\textrm{g}$}
\label{my-label}
\begin{tabular}{|c|c|c|}
\hline
      & θL      & θR      \\ \hline
赤 一次  & 347°00' & 301°33' \\ \hline
赤 二次  & 375°28' & 274°16' \\ \hline
緑 一次  & 341°58' & 306°27' \\ \hline
緑 二次  & 362°08' & 286°56' \\ \hline
薄青 一次 & 340°56' & 307°29' \\ \hline
薄青 二次 & 359°43' & 289°14' \\ \hline
濃青 一次 & 340°29' & 307°56' \\ \hline
濃青 二次 & 358°37' & 290°16' \\ \hline
紫 一次  & 339°23' & 308°50' \\ \hline
紫 二次  & 356°22' & 292°26' \\ \hline
\end{tabular}
\end{table}

\section{課題}

\subsection{(1)}

まず、次の式からそれぞれの測定での$N_i$を求める。

\begin{equation}
    N_i = \frac{\sin{\theta_m}}{m\lambda_i}
\end{equation}

この時の不確かさは次の式を用いて算出した。ここでは$\Delta \theta_m = 0.01$,$\Delta \lambda_i = 0.1$として計算した。

\begin{equation}
    \Delta N_i = N_i\cdot\sqrt{(\frac{\cos{\theta_m}}{\sin{\theta_m}}^2) + (\frac{\Delta\lambda_i}{\lambda_i})^2}
\end{equation}

上式の計算結果は次の表のようになった。

\begin{table}[H]
\centering
\caption{測定結果から求められた格子定数とその不確かさ}
\label{my-label}
\begin{tabular}{|l|l|l|l|l|}
\hline
         & D1 橙 一次 & D1 橙 一次 & D2 橙 一次 & D2 橙 一次 \\ \hline
N\_i /mm & $599 \pm 15$     & $599 \pm 6$     & $599 \pm 16$     & $600 \pm 6$     \\ \hline
\end{tabular}
\end{table}

\subsection{(2)}

スペクトル線の波長を求めるために次式を用いた。

\begin{equation}
    \lambda = \frac{\sin{\theta_m}}{mN}
\end{equation}

また、この時の不確かさは次の式を用いて計算した。

\begin{equation}
    \Delta \lambda = \lambda\cdot\sqrt{(\frac{\cos{\theta_m}}{\sin{\theta_m}}^2) + (\frac{\Delta N}{N})^2}
\end{equation}


\begin{table}[]
\centering
\caption{水素のスペクトルの波長}
\label{my-label}
\begin{tabular}{|c|c|c|c|c|c|c|}
\hline
          & 赤 一次  & 赤 二次  & 緑 一次  & 緑 二次  & 青 一次  & 青 二次  \\ \hline
$\lambda/10^{-7}$mm & $6.374\pm0.002$ & $6.567\pm0.001$ & $4.861\pm0.001$ & $4.863\pm0.001$ & $4.344\pm0.002$ & $4.340\pm0.001$ \\ \hline
\end{tabular}
\end{table}


\begin{table}[H]
    \centering
    \caption{カドミウムのスペクトルの波長-1}
    \label{my-label}
    \begin{tabular}{|c|c|c|c|c|c|c|}
    \hline
             & 赤 一次  & 赤 二次  & 緑 一次  & 緑 二次  & 薄青 一次 & 薄青 二次 \\ \hline
    $\lambda/10^{-7}$mm & $6.439 \pm0.001$ & $6.440\pm0.002$ & $5.08\pm0.001$ & $5.085\pm0.002$ & $4.796\pm0.001$ & $4.809\pm0.002$ \\ \hline
    \end{tabular}
    \end{table}

    \begin{table}[H]
        \centering
        \caption{カドミウムのスペクトルの波長-2}
        \label{my-label}
        \begin{tabular}{|c|c|c|c|c|}
        \hline
                 & 濃青 一次 & 濃青 二次 & 紫 一時  & 紫 二次  \\ \hline
        $\lambda/10^{-7}$mm & $4.671\pm0.002$ & $4.681\pm0.001$ & $4.391\pm0.002$ & $4.412\pm0.001$ \\ \hline
        \end{tabular}
        \end{table}

文献値は次のようなものであった。

\begin{table}[H]
    \centering
    \caption{水素のスペクトルの波長の文献値}
    \label{my-label}
    \begin{tabular}{|c|c|c|c|c|c|c|}
    \hline
              & 赤 一次           & 赤 二次          & 緑 一次           & 緑 二次          & 青 一次          & 青 二次          \\ \hline
    $\lambda$文献値$ /10^{-7}$mm & \multicolumn{2}{c|}{6.5627110} & \multicolumn{2}{c|}{4.8612870} & \multicolumn{2}{c|}{4.340472} \\ \hline
    \end{tabular}
    \end{table}

\begin{landscape}
\begin{table}[H]
    \centering
    \caption{カドミウムのスペクトルの波長の文献値}
    \label{my-label}
    \begin{tabular}{|c|c|c|c|c|c|c|c|c|c|c|}
    \hline
                 & 赤 一次           & 赤 二次          & 緑 一次           & 緑 二次          & 薄青 一次          & 薄青 次          & 濃青 一次          & 濃青 二次         & 紫 一時          & 紫 二次         \\ \hline
    $\lambda$文献値$ /10^{-7}$mm & \multicolumn{2}{c|}{6.4402480} & \multicolumn{2}{c|}{5.0872379} & \multicolumn{2}{c|}{4.8012521} & \multicolumn{2}{c|}{4.6794581} & \multicolumn{2}{c|}{4.41563} \\ \hline
    \end{tabular}
    \end{table}
\end{landscape}

\subsection{(3)}

受験期には光の波長は「赤橙黄緑青紫」の順番に波長が長い順に並んでいると教わったが今回の実験結果から実際に、光の波長は先ほどの順番と同じように波長が長い順に並んでいるという関係を確認することができた。

\section{考察}

\subsection{課題(1)}

課題(1)で求めた格子定数は、基礎科学実験AのWEBサイトより600であるということである。従って表4より、測定結果から得られた格子定数はどの値もこの格子定数600という値を不確かさの範囲の中に収めている。よって、今回の実験において使用した格子定数の値は非常に良い精度で求めることができたと考察する。

\subsection{課題(2)}

\subsubsection{水素のスペクトルの波長について}

今回の実験では実験結果から求められた値とその不確かさの範囲の中に文献値が収まったスペクトルはなかった。しかし、この実験結果は不確かさの桁である小数第3位と同じ桁での誤差なので悪くない精度で値を出すことができたと考察する。

\subsubsection{カドミウムのスペクトルの波長について}

今回の実験では実験結果から求められた値とその不確かさの範囲の中に文献値が収まったスペクトルは赤色の2次だけであった。また、水素のスペクトルの波長と同じくらい正確に測定できたスペクトルが大半であったが、薄青色のスペクトルはかなりズレていると考察できる。

\subsection{全般について}
以上の結果となったがこのような結果となった原因としては二つあると考察する。
1つ目は実験手順の中に垂直にするというものがあったが、これは光源となる管と実験装置を最初から垂直に固定してあるべきだと考察する。なぜなら、管の根本部分のみ交換可能とすると、垂直になっていないがために微調整で必要とされる時間をかなり削減できる上に、わずかな誤差の調整に時間をかけることができるのでより、正確な精度で測定できると考察するからである。
2つ目はやはり、焦りである。自分たちは最初の微調整で大幅におくれてしまったため、かなり実験中に焦りが生じてしまった。したがって、慎重に実験することでもまた、より高い精度で測定できると考察する。

\begin{thebibliography}{99}
    \bibitem{UEC} 共通教育部自然科学部会(物理)、『基礎科学実験A(物理学実験)平成29年(2017年)版』
\end{thebibliography}

基礎科学実験Aの資料のページ
url{http://physics.e-one.uec.ac.jp/theme.html}

\end{document}