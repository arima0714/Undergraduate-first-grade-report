\documentclass{jsarticle}
\usepackage{booktabs}
\usepackage{here}
\usepackage{url}
\begin{document}
\title{光のスペクトル}
\author{有馬海人}
\maketitle

\section{実験の目的}

光源にNaランプやHgランプなどを用い、回折格子分光計を使って、それぞれの原子に特有なスペクトル線を観測し、その波長を用いる。

\section{実験の原理}

\subsection{スペクトル線について}

量子理論によれば、水素原子のエネルギーは次のとびとびの値しかとることができない。

\begin{equation}
    E_n = -\frac{hcR}{n^2} (n = 1,2,3,\cdots)
\end{equation}

ここで$hcR$は定数である。$h$はプランク定数,$c$は真空中の光速度で、$R$はリュードベリ定数と呼ばれる。整数$n$で決められる原子状態及びエネルギーの値をエネルギー準位という。電気放電などによって高いエネルギー準位$n_1$に上げられた原子がより低いエネルギー準位$n_2$へ状態を変えるときに光が放出され、その周波数$\nu$と波長$\lambda$は次の式から決定される。

\begin{equation}
    h\nu = E_{n_1} - E_{n_2} 及び \frac{1}{\lambda} = \frac{\nu}{c} = \frac{1}{hc} (E_{n_1} - E_{n_2})
\end{equation}

水素原子の可視域のスペクトル線は$n > 2$の順位から$n = 2$の順位へ遷移するときに放出され、

\begin{equation}
    \frac{1}{\lambda} = \frac{1}{hc}(E_{n_1} - E_{n_2}) = R(\frac{1}{2^n}{2^n}) (n = 3,4,5,\cdots)
\end{equation}

から計算される。

\subsection{回折格子の原理}

スリットを狭い間隔で平行に並べたものを回折格子という。特定の波長の平行光線を回折格子の面に垂直に当てると、回折格子を通った光はいくつかの方向に現れる。入射方向に出てくる光線を0次,それより角度のます方向に順次1次,2次,$\cdots$の回折光という。入射光の方向と回折光の方向のなす角を回折角という。m次回折角$\theta_m$は次の式で与えられる。

\begin{equation}
    d\sin{\theta_m} = m\lambda または \sin{\theta_m} = m\lambda N (m = \pm 1,\pm 2, \cdots)
\end{equation}

ここで$\lambda$は光の波長,$d$は回折格子の格子の間隔,$N = 1/d$はdの逆数で単位長さあたりの格子の数である。回折格子の特性を表す時は$d$よりも$N$を用いるのが普通である。上式は隣接するスリットを通った光が強め合う条件である。色々な波長を含む光を回折格子に当てると回折光は分かれ、入射した光のスペクトルが観測される。次数が同じ回折行を調べると波長の長い赤いスペクトル線の方が波長の短い青いスペクトル線よりも回折角が大きいことに注意する。

\begin{equation}
    実験の原理
\end{equation}


\section{実験の方法}

実験の方法

\section{測定結果}

測定結果

\section{課題}

課題

\section{考察}

考察






\begin{thebibliography}{99}
    \bibitem{UEC} 共通教育部自然科学部会(物理)、『基礎科学実験A(物理学実験)平成29年(2017年)版』
\end{thebibliography}

\end{document}