\documentclass{jsarticle}
\usepackage{booktabs}
\usepackage{here}
\begin{document}
\title{二次元の等電位線}
\author{有馬海人}
\maketitle

\section{実験の目的}
導電性のカーボン紙を利用し、電位を直接観測することにより等電位線と電気力線を実際に描く。無限に広い平面上での理論と実験結果を比較することにより、等電位線と電気力線の間にどの様な関係があるのか考察する。

\section{実験の原理}
導電性カーボン紙に電極を塗布して、カーボン紙上の等電位点点をたどることによって等電位線を描く。実際に観測しているのは、カーボン紙上に生じる定常電流の場における等電位線と電気力線である。子の等電位線と電留線の2つの曲線群は互いに直行する。同様な関係が次の表に示すように。静電場における等電位線と電留線の2つの曲線群は互いに直交する。同様な関係が次の表に示すように、静電場における等電位線と電流線、保存力場における等ポテンシャル線と力線、定常的な流れの場における等速度ポテンシャル線と流線、等々に対しても成り立つ。\\
\par いくつかの簡単な場合には定常電流の場(または静電場)における等電位線と電流線は簡単な式で表される。\\
\par 無限に広い導電性平面上に、距離Dを隔てて2つの円形電極がある場合を考える。電極の一方の電位を0、他方の電位を$V_0$とする。この時電位0の電極の中心から距離r、電位$V_0$の電極の中心から距離rの点Pにおける電位Vは次の式で与えられる。\\


\begin{equation}
\label{4.1}
	v = \frac{V_0}{2}\frac{\log(\frac{R}{D})}{\log(\frac{R}{D})}
\end{equation}

したがって等電位線は$r'/r = const.$を満たす曲線、すなわちアポロニウスの円となる。また、電気力線または電流線は両電極の中心を通る円となる。\\
\par 次に無限に広い導電性平面状に、距離Dを隔てて2つの円形電極がある場合を考える。電極の一方の電位を0、他方の電位を$V_0$とする。この時の電位0の電極の中心から距離$r$、電位$V_0$の電極の中心から距離$r'$の電極の中心から距離rの点Pにおける電位Vは次の式で与えられる。

\begin{equation}
\label{4.2}
	v = \frac{V_0}{2}\frac{\log(\frac{Rr'}{Dr})}{\log(\frac{R}{D})}
\end{equation}

したがって等電位線は$r'/r = const$を満たす曲線、すなわちアポロニウスの円となる。また、電気力線または電流線は両電極の中心を通る円となる。
次に無限に広い一様な電場の中に半径$R$の導体または絶縁体がある場合を考える。
(1) 無限に広い一様な$y$方向の電場の中に半径$R$の導体の円板がある場合

\begin{equation}
\label{4.4}
		y - \frac{R^2y}{x^2+y^2} = const
\end{equation}

\begin{equation}
\label{4.5}
		x + \frac{R^2x}{y^2+x^2} = const
\end{equation}

(2) 無限に広い一様な$y$方向の電場の中に半径$R$の絶縁体の円板がある場合

\begin{equation}
\label{4.6}
	y + \frac{R^2y}{x^2+y^2} = const
\end{equation}
	
\begin{equation}
\label{4.7}
	x - \frac{R^2x}{y^2+x^2} = const
\end{equation}

(3) 無限に広い一様な$x$方向の電場の中に半径$R$の導体の円板がある場合

\begin{equation}
\label{4.8}
		x - \frac{R^2x}{x^2+y^2} = const
\end{equation}

\begin{equation}
\label{4.9}
		y + \frac{R^2y}{y^2+x^2} = const
\end{equation}

(4) 無限に広い一様な$y$方向の電場の中に半径$R$の導体の円板がある場合

\begin{equation}
\label{4.10}
		x + \frac{R^2x}{x^2+y^2} = const
\end{equation}

\begin{equation}
\label{4.11}
		y - \frac{R^2y}{y^2+x^2} = const
\end{equation}




\section{実験の方法}

\subsection{2つの円形電極の周りの等電位線を求める}
\begin{enumerate}
	\item 長方形のカーボン紙を短辺の長さに合わせて正方形にした。
	\item 正方形のカーボン紙に原点間が4cmの半径4cmの円と半径3.5cmの円をそれぞれ2つずつ書いた。
	\item 2円で構成された2つのドーナツ状の部分に銀ペーストを塗布した。
	\item 導体のどこに触れても等電位であることを確認した。
	\item 1Vから5Vまで1Vずつ等電位線を描いた。

\end{enumerate}

\subsection{一様な電場の中の円形導体のまわりの等電位線と電気力線を求める。}
\begin{enumerate}
	\item 新しい長方形のカーボン紙の中央に半径4cmの円と半径3.5cmの円を書き、その間のドーナツ状の部分に銀ペーストを塗布した。
	\item また、長方形の短辺の両端に幅5mmで導電塗料を塗布した。
	\item 導体のどこに触れても等電位であることを確認した。	\item 1Vから5Vまで1Vずつ等電位線を描いた。
	\item 中央の円形導電体を切り落とした。
	\item 短辺の両端に塗布した導電塗料の部分を両方切り落とした。
	\item 切り落とした方でない両端に導電塗料を幅5mmで塗布した。
	\item 3Vから5Vまで1Vずつ等電位線を描いた。
\end{enumerate}

\section{測定結果と課題}

\ref{4.2}式に$V$の値を代入して算出した$r'/r$の結果は下表のようになった。

\begin{table}[H]
	\centering
	\caption{電圧とr'/rの表}
	\label{my-label}
	\begin{tabular}{|l|l|}
	\hline
	$V/\textrm V$ & r'/r \\ \hline
	1   & 4.00 \\ \hline
	2   & 2.00 \\ \hline
	3   & 1.00 \\ \hline
	4   & 0.50 \\ \hline
	5   & 0.25 \\ \hline
	\end{tabular}
	\end{table}

両電極の中心A,Bの間を$1:(r'/r)$に内分する点C,Dの座標は下表のようになった。

\begin{table}[H]
	\centering
	\caption{電圧と点Cと点Dの座標}
	\label{my-label}
	\begin{tabular}{|l|l|l|}
	\hline
	V/V & C /mm            & D /mm           \\ \hline
	1   & -3.602213614 & -9.99385485  \\ \hline
	2   & -2.00614869  & -17.9448314  \\ \hline
	3   & -0.010379898 & -3468.242338 \\ \hline
	4   & 1.987688443  & 18.11149033  \\ \hline
	5   & 3.588908942  & 10.03090371  \\ \hline
	\end{tabular}
	\end{table}

実験1のレポート課題(iii)の結果はレポートの末尾に添付した。\\
\par 実験2レポート課題(4)の座標を求める際には\ref{4.4}式を変形した次式を利用した。

\begin{equation}
\label{4.8}
	x = \pm\sqrt{\frac{y(-y^2+cy+R^2)}{y-c}}
\end{equation}

実験2レポート課題(5)の座標を求める際には\ref{4.10}式を変形した次式を用いた。
\begin{equation}
\label{4.8}
	y = \pm\sqrt{\frac{x(x^2-cx+R^2)}{c-x}}
\end{equation}

課題(4)(5)において作成したグラフ及び計算図はレポートの末尾に添付した。

\section{考察}

\subsection{実験1について}
%\subsubsection{実験課題(3)について}
等電位線とは直行することが3本の等電位線から確認することができた。したがって、実際にカーボン紙の端は絶縁体であることも分かったと考察する。赤の実戦で描いた計算上の図とは線分点A点Bでは非常に近い値となっている。また、電位差が1V、5Vの場合の等電位線は計算上のものと近い外形となっている。以上から、2円に等電位線が近くなればなるほど計算上の値に近くのであると考察する。

\subsection{実験2について}
作成したグラフより、中心の円から遠ざかるほど計算図と測定図の違いがなくなってきている。このことから、円に遠ざかるほど円による影響が小さくなっていくため、計算値と実験値に違いがなくなっていくのであると考察する。また、実験値の等電位線と電気力線を見ると、これらは無限に広い平面上だけでなく実際に有限の平面であっても直交している。このことから、 等電位線のベクトルと電気力線のベクトルの内積は0であると考察する。

\subsection{実験をスムーズに行うために}
今回の実験で私は銀ペーストを塗布した部分を等電位にするのに時間がかかってしまった。このようなことをなくすために次に挙げることを行うと良いと考察する。
1つ目は銀ペーストに含まれている銀の濃度を挙げることであると考察する。銀がペーストの内部でまとまってしまい、様々な場所で等電位にするのには重ね塗りが非常に多くの数をする必要が出てしまい、大幅に時間を使用してしまっていた。しかし、これを行うことによってのがすぐに銀成分が全体に行き渡りやすくなるので、電位が安定しやすくなると考察する。
2つ目は銀ペーストの塗り方であると考察する。これをハケといらない紙などを利用することで満遍なく漏れ無く、導電体を作ることができるからであると考察する。出てはいけない部分を不要な紙で隠すことで安心して銀ペーストを塗布することができる。また、ハケを使うことで先の細い筆で長い時間使わずに効率的に銀ペーストの塗布を行うことができるからであると考察する。
そして、3つ目はドライヤーの最高温度を上げることである。なんども塗り直していたが、気がつけば電位が安定するようになったということが発生していたことから、このようなことが発生していたのには乾燥不足があったのだと考察する。したがって、ドライヤーの最高温度を挙げることでこの乾燥不足が起きることが少なくなり、実際の実験以外で時間の効率化を図ることができると考察する。

\begin{thebibliography}{99}
	\bibitem{UEC} 共通教育部自然科学部会(物理)、『基礎科学実験A(物理学実験)平成29年(2017年)版』
\end{thebibliography}
\end{document}
