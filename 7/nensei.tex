\documentclass{jsarticle}
\usepackage{booktabs}
\usepackage{here}
\usepackage{url}
\begin{document}
\title{粘性率}
\author{\empty}
\date{\empty}
\maketitle

\section{実験の目的}

細いガラス管の中に液体を吸い上げ、その降下する速さを測定することによって、液体に特徴的な物理定数である粘性率と表面張力を求める。

\section{実験の原理}

\subsection{表面張力}

液体の中の分子は周りの分子から引力を受けている。この力は、ごく近寄れば大きいが、ちょっと離れるとたちまち0となる。この力の及ぶ距離は$10^-9$m程度までである。さて、体内部にある分子は四方八方から一様な引力を受けるが、分子が液体表面から$d$以内の距離に入ると、分子に作用する力は不均衡となり、分子は内部に引っ張り込まれようとする。のことは液体の表面が自分から縮もうとする性質を持っていることを意味する。言い換えれば、液体の自由表面積を広げてやるには仕事が必要である。液体の自由表面を単位面積だけ加させるのに必要な仕事を表面張力という。少量の水が自然には球状の水滴になるには、体積が一定ならば、表面積の一番小さい形は球であるからである。水の量が多くなると表面張より重力の作用がはるかに大きくなるが、無重力空間では大きな水の球を作ることができる。\\
\par 針金の枠ABCD上を、BCと平行に保ったまま摩擦なく滑る針金B'C'を考える。長方形B'C'CBに液体の膜を張ると、針金B'C'には膜の面積を小さくしようとする力が作用する。この力の大きさを$F$とする。針金を$\Delta x$移動させるとしよう。膜には表と裏のあることを考えると液体の表面積の増加は$2a\Delta x$である。必要な仕事$F\Delta x$は面積の増加分に比例する。

\begin{equation}
     F\Delta x = \gamma 2a\Delta x
\end{equation}

比例定数$\gamma$が表面張力である。2aは液体の境界線の長さ$l$であるから、次の関係式が得られる。

\begin{equation}
    F = \gamma l
\end{equation}

つまり表面張力は境界線の単長さ当たり作用する力に等しい。以上から表面張力の単位は$\textrm{j}/\textrm{m}^2$とも$\textrm{N}/textrm{m}$とも表すことができる。

\subsection{粘性率}

粘性とは流体中の摩擦現象である。流体の隣り合った部分が異なる速度で流れている場合に、速度の差を無くそうとする力が作用する現象である。極低温の超流動状態の液体ヘリウムをのぞいて全ての液体、気体は粘性を有する。\\
\par 粘性のある流体が容器の壁に摂食する部分では、壁に相対的な流体の速度は0である。\\
\par 2枚の水平な板の間の流体を考えよう。固定された下の板に対しては上の板は一定の速度$v$で運動しているとする。このとき下の板に接する流体の速度は0、上の板に接する流体の速度は$v$である。この時下の板は流れの方向に、上の板は流れと逆向きに力を受ける。板の面積$S$が受ける力の大きさ$F$は面積$S$と速度勾配$v/h$に比例する。つまり単位面積あたりの力は

\begin{equation}
    \frac{F}{S} = \eta\frac{v}{h}
\end{equation}

と表される。この比例定数$\eta$を粘性率または粘性係数という。流体を板に平行な面で仮想的に分けて考えると、この面を通して上下の液体には上式で与えられるずれ応力が作用していることになる。なお、粘性係数の単位は、$\textrm{Pa}\cdot\textrm{s}$である。

\subsection{測定の原理}

円形断面を持つ管の中を流体が乱れなく流れている場合において管壁に摂食する部分では流速は0であるので、管が細くなると流量は急速に小さくなる。管の半径を$a$,長さを$l$,両端での圧力差を$p_1 - p_2$とすると時間$t$の間に流れる流体の体積$V$は面積

\begin{equation}
    V = \frac{\pi a^4(P_1 - P_2)}{8\eta l}t
\end{equation}

であることが知られている。これをポアズイユの法則という。管内の平均流速%v%は単位時間の流量$4V/t$を断面積で割って得られる。

\begin{equation}
\label{7.6}
    v = \frac{V}{\pi a^2 t} = \frac{a^2(p_1 - p_2)}{8\eta l}
\end{equation}

内半径$a$の細いガラス管を鉛直から角$\theta$だけ傾け、下端を液面に触れさせる。このガラス管の中に液体を吸い上げ、途中に切れ目のない長い液柱を作る。管内を大気に戻すと液柱の上端がゆっくりと降下する。液体の密度$\rho$,液柱の長さを$l$とすると液柱の質量は$2\pi a\gamma$である。したがって\ref{7.6}式の圧力差に

\begin{equation}
    p_1 - p_2 = \frac{\pi a^2 l \rho g \cos{\theta} - 2\pi a \gamma}{\pi a^2} = (\rho g \cos{\theta} - \frac{2\gamma}{al})l
\end{equation}

を代入して次の式を得る。

\begin{equation}
\label{7.8}
    v = \frac{a^2}{8\eta}(\rho g \cos{\theta} - \frac{2\gamma}{al})
\end{equation}

これが液頭の降下する速さである。降下すると液柱の長さは短くなるので次第に降下速度は遅くなり、ついには止まる。止まった時$(v = 0)$の液柱の長さは$l_0$は

\begin{equation}
    l_0 = \frac{2\gamma}{a\rho g \cos{\theta}}
\end{equation}

である。ガラス管内で液体が表面張力によって引き上げられる現象を毛細現象という。毛細現象が認められるような細い管を毛細管と呼ぶ。\\

\par 毛細管内に吸い上げた液柱の上端の降下速度$v$を縦軸に、液柱の長さの逆数$1/l$を横軸に取って\ref{7.8}をグラフに表したとき、直線が縦軸と交わる点を$v_0$、横軸と交わる点の逆数を$l_0$とすれば、$v_0$,$l_0$はそれぞれ粘性率$\eta$,表面張力$\gamma$と次の関係にある。

\begin{equation}
    \eta = \frac{a^2}{8v_0}\rho g \cos{\theta}
\end{equation}

\begin{equation}
    \gamma = \frac{1}{2}al_0\rho g \cos{\theta}
\end{equation}

\section{実験の方法}

\begin{enumerate}
\item 使用する毛細管の番号と半径をメモした
\item 毛細管に付いていた標線に上から番号を順に割り振り、下端から各標線までの長さ$L_1,L_2,L_3,\cdots$を測定した。
\item 試料の溶液をシャーレに取り、鉛直方向から$50^\circ から 70^\circ$傾けて毛細管を立てた。そして備え付けの分度器で角度を測定した。
\item 試料液体の密度をヘアの装置を使って測定した。
\item 気温を測定した。
\item 毛細管に泡などが生じたり、液柱が分断されないようにゆっくりと吸い上げた。
\item 毛細管内を大気圧に戻して、液頭が標線を次々と横切っていく時刻を計測、記録した。
\item これらを各試料液体について3回繰り返した。
\end{enumerate}

\section{測定結果}
\begin{center}
\par $L_i$:下端から標線までの距離
\par $\Delta L_i$:区間の長さ
\par $l_i = \frac{L_i + L_{i+1}}{2}$:区間の中点
\end{center}

\begin{table}[H]
\centering
\caption{毛細管の測定値}
\label{my-label}
\begin{tabular}{|l|l|l|}
\hline
\multicolumn{3}{|l|}{PF-04,0.01685 cm} \\ \hline
$L_i$ /cm   & $\Delta L_i$   & $1/l_i$    \\ \hline
83.97    &            &         \\ \hline
75.7     & 8.27       & 0.0125  \\ \hline
68.62    & 7.08       & 0.0139  \\ \hline
60.81    & 7.81       & 0.0155  \\ \hline
54.18    & 6.63       & 0.0174  \\ \hline
46.57    & 7.61       & 0.0199  \\ \hline
38.75    & 7.82       & 0.0234  \\ \hline
32       & 6.75       & 0.0283  \\ \hline
26.23    & 5.77       & 0.0343  \\ \hline
20.08    & 6.15       & 0.0432  \\ \hline
16.31    & 3.77       & 0.0550  \\ \hline
12.45    & 3.86       & 0.0695  \\ \hline
9.09     & 3.36       & 0.0929  \\ \hline
6.28     & 2.81       & 0.1301  \\ \hline
3.14     & 3.14       & 0.2123  \\ \hline
\end{tabular}
\end{table}

ヘアの装置によって測定した結果はエタノール,混合溶液それぞれ、下記の2表のようになった。

\begin{table}[H]
\centering
\caption{ヘアの装置による測定結果1}
\label{my-label}
\begin{tabular}{|l|l|l|}
\hline
  & 水     & エタノール \\ \hline
前 & 30.95 & 39.20 \\ \hline
後 & 5.76  & 7.22  \\ \hline
差 & 25.19 & 31.98 \\ \hline
\end{tabular}
\end{table}

\begin{table}[H]
\centering
\caption{ヘアの装置による測定結果2}
\label{my-label}
\begin{tabular}{|l|l|l|}
\hline
  & 水     & 混合溶液  \\ \hline
前 & 36.60 & 41.08 \\ \hline
後 & 11.61 & 12.46 \\ \hline
差 & 24.99 & 28.62 \\ \hline
\end{tabular}
\end{table}

上の表の値を用いて密度を計算した結果は下記のようになった。また、式は次のものを用いた。

\begin{equation}
    \rho = \frac{\textrm{h}_1 - \textrm{h}_2}{\textrm{h'}_1 - \textrm{h'}_2}
\end{equation}

\begin{table}[H]
\centering
\caption{液体試料の密度}
\label{my-label}
\begin{tabular}{|l|l|l|l|}
\hline
                & 水 & エタノール  & 混合溶液   \\ \hline
密度 rho /g・cm\_1 & 1 & 0.7877 & 0.8732 \\ \hline
\end{tabular}
\end{table}

\begin{center}
\par $T_i$:標線通過時刻
\par $\Delta T_i = T_{i+1} - T_i$:時間間隔
\par $v_i = \frac{\Delta L_i}{\Delta T_i}$
\end{center}

\begin{table}[H]
\centering
\caption{水の測定結果1}
\label{my-label}
\begin{tabular}{|l|l|l|}
\hline
\multicolumn{3}{|l|}{水,18.6℃,theta = 60.3,ls = 16.36} \\ \hline
$T_i $/s   & $\Delta T_i $ /s  & $v_i = \frac{\Delta L_i}{\Delta T_i}$ /cm$\cdot \textrm{s}^{-1}$   \\ \hline
6.38           & 6.38           & 1.296         \\ \hline
11.73          & 5.35           & 1.323         \\ \hline
17.85          & 6.12           & 1.276         \\ \hline
23.23          & 5.38           & 1.232         \\ \hline
29.98          & 6.75           & 1.127         \\ \hline
37.73          & 7.75           & 1.009         \\ \hline
45.45          & 7.72           & 0.874         \\ \hline
53.98          & 8.53           & 0.676         \\ \hline
66.11          & 12.13          & 0.507         \\ \hline
\end{tabular}
\end{table}

\begin{table}[H]
\centering
\caption{水の測定結果2}
\label{my-label}
\begin{tabular}{|l|l|l|}
\hline
\multicolumn{3}{|l|}{水,18.9℃,theta = 60.3,ls = 16.56} \\ \hline
$T_i $/s   & $\Delta T_i $ /s  & $v_i = \frac{\Delta L_i}{\Delta T_i}$ /cm$\cdot \textrm{s}^{-1}$   \\ \hline
6.01           & 6.01           & 1.376         \\ \hline
11.35          & 5.34           & 1.325         \\ \hline
17.5           & 6.15           & 1.269         \\ \hline
22.99          & 5.49           & 1.207         \\ \hline
29.7           & 6.71           & 1.134         \\ \hline
37.32          & 7.62           & 1.026         \\ \hline
45.01          & 7.69           & 0.877         \\ \hline
53.53          & 8.52           & 0.677         \\ \hline
65.66          & 12.13          & 0.507         \\ \hline
\end{tabular}
\end{table}

\begin{table}[H]
\centering
\caption{水の測定結果3}
\label{my-label}
\begin{tabular}{|l|l|l|}
\hline
\multicolumn{3}{|l|}{水,18.9℃,theta = 60.3,ls = 16.51} \\ \hline
$T_i $/s   & $\Delta T_i $ /s  & $v_i = \frac{\Delta L_i}{\Delta T_i}$ /cm$\cdot \textrm{s}^{-1}$   \\ \hline
6.01           & 6.01           & 1.376         \\ \hline
11.36          & 5.35           & 1.323         \\ \hline
17.56          & 6.2            & 1.259         \\ \hline
23.01          & 5.45           & 1.216         \\ \hline
29.65          & 6.64           & 1.146         \\ \hline
37.32          & 7.67           & 1.019         \\ \hline
44.95          & 7.63           & 0.884         \\ \hline
53.4           & 8.45           & 0.682         \\ \hline
65.48          & 12.08          & 0.509         \\ \hline
\end{tabular}
\end{table}

\begin{table}[H]
\centering
\caption{エタノールの測定結果1}
\label{my-label}
\begin{tabular}{|l|l|l|}
\hline
\multicolumn{3}{|l|}{エタノール,19.1℃,theta = 60.2,ls = 6.51} \\ \hline
$T_i $/s   & $\Delta T_i $ /s  & $v_i = \frac{\Delta L_i}{\Delta T_i}$ /cm$\cdot \textrm{s}^{-1}$   \\ \hline
7.68            & 7.68            & 1.076          \\ \hline
14.55           & 6.89            & 1.030          \\ \hline
22.2            & 7.65            & 1.020          \\ \hline
28.77           & 6.57            & 1.009          \\ \hline
36.57           & 7.8             & 0.975          \\ \hline
44.96           & 8.39            & 0.932          \\ \hline
52.3            & 7.34            & 0.919          \\ \hline
59.3            & 7               & 0.824          \\ \hline
66.36           & 7.06            & 0.871          \\ \hline
73.96           & 7.6             & 0.496          \\ \hline
81.98           & 8.02            & 0.481          \\ \hline
92.55           & 10.57           & 0.317          \\ \hline
\end{tabular}
\end{table}

\begin{table}[H]
\centering
\caption{エタノールの測定結果2}
\label{my-label}
\begin{tabular}{|l|l|l|}
\hline
\multicolumn{3}{|l|}{エタノール,19.3℃,theta = 60.2,ls = 6.90} \\ \hline
$T_i $/s   & $\Delta T_i $ /s  & $v_i = \frac{\Delta L_i}{\Delta T_i}$ /cm$\cdot \textrm{s}^{-1}$   \\ \hline
7.8             & 7.8            & 1.0602           \\ \hline
14.52           & 6.72           & 1.0535           \\ \hline
22.14           & 7.62           & 1.0249           \\ \hline
28.54           & 6.4            & 1.0359           \\ \hline
36.14           & 7.6            & 1.0013           \\ \hline
44.35           & 8.21           & 0.9524           \\ \hline
51.52           & 7.17           & 0.9414           \\ \hline
58.22           & 6.7            & 0.8611           \\ \hline
64.94           & 6.72           & 0.9151           \\ \hline
71.66           & 6.72           & 0.5610           \\ \hline
78.7            & 7.04           & 0.5482           \\ \hline
87.48           & 8.78           & 0.3826           \\ \hline
\end{tabular}
\end{table}

\begin{table}[H]
\centering
\caption{エタノールの測定結果3}
\label{my-label}
\begin{tabular}{|l|l|l|}
\hline
\multicolumn{3}{|l|}{エタノール,19.4℃,theta = 60.2,ls = 6.41} \\ \hline
$T_i $/s   & $\Delta T_i $ /s  & $v_i = \frac{\Delta L_i}{\Delta T_i}$ /cm$\cdot \textrm{s}^{-1}$   \\ \hline
7.78            & 7.78           & 1.062982005           \\ \hline
14.26           & 6.48           & 1.092592593           \\ \hline
21.81           & 7.55           & 1.034437086           \\ \hline
28.13           & 6.32           & 1.049050633           \\ \hline
35.68           & 7.55           & 1.00794702            \\ \hline
43.56           & 7.88           & 0.992385787           \\ \hline
50.81           & 7.25           & 0.931034483           \\ \hline
57.33           & 6.52           & 0.884969325           \\ \hline
63.93           & 6.6            & 0.931818182           \\ \hline
70.56           & 6.63           & 0.568627451           \\ \hline
77.14           & 6.58           & 0.58662614            \\ \hline
85.81           & 8.67           & 0.387543253           \\ \hline
\end{tabular}
\end{table}

\begin{table}[H]
\centering
\caption{混合溶液の測定結果1}
\label{my-label}
\begin{tabular}{|l|l|l|}
\hline
\multicolumn{3}{|l|}{混合,20.8℃,theta = 60.2,ls = 6.65} \\ \hline
$T_i $/s   & $\Delta T_i $ /s  & $v_i = \frac{\Delta L_i}{\Delta T_i}$ /cm$\cdot \textrm{s}^{-1}$   \\ \hline
11.97           & 11.97         & 0.690893901         \\ \hline
22.12           & 10.15         & 0.697536946         \\ \hline
33.78           & 11.66         & 0.669811321         \\ \hline
43.56           & 9.78          & 0.67791411          \\ \hline
55.12           & 11.56         & 0.658304498         \\ \hline
67.32           & 12.2          & 0.640983607         \\ \hline
78.32           & 11            & 0.613636364         \\ \hline
88.52           & 10.2          & 0.565686275         \\ \hline
98.7            & 10.18         & 0.604125737         \\ \hline
108.7           & 10            & 0.377               \\ \hline
119.35          & 10.65         & 0.362441315         \\ \hline
133.16          & 13.81         & 0.243301955         \\ \hline
\end{tabular}
\end{table}

\begin{table}[H]
\centering
\caption{混合溶液の測定結果2}
\label{my-label}
\begin{tabular}{|l|l|l|}
\hline
\multicolumn{3}{|l|}{混合,20.9℃,theta = 60.2,ls = 6.74} \\ \hline
$T_i $/s   & $\Delta T_i $ /s  & $v_i = \frac{\Delta L_i}{\Delta T_i}$ /cm$\cdot \textrm{s}^{-1}$   \\ \hline
10.28           & 10.28         & 0.688715953         \\ \hline
20.44           & 10.16         & 0.768700787         \\ \hline
31.89           & 11.4          & 0.579039301         \\ \hline
41.66           & 9.77          & 0.778915046         \\ \hline
53.32           & 11.66         & 0.670668954         \\ \hline
65.39           & 12.07         & 0.55923778          \\ \hline
76.4            & 11.01         & 0.524069028         \\ \hline
86.57           & 10.17         & 0.604719764         \\ \hline
96.8            & 10.23         & 0.368523949         \\ \hline
107.02          & 10.22         & 0.377690802         \\ \hline
117.64          & 10.62         & 0.316384181         \\ \hline
131.73          & 14.09         & 0.199432221         \\ \hline
\end{tabular}
\end{table}

\begin{table}[H]
\centering
\caption{混合溶液の測定結果3}
\label{my-label}
\begin{tabular}{|l|l|l|}
\hline
\multicolumn{3}{|l|}{混合,21.1℃,theta = 60.2,ls = 6.69} \\ \hline
$T_i $/s   & $\Delta T_i $ /s  & $v_i = \frac{\Delta L_i}{\Delta T_i}$ /cm$\cdot \textrm{s}^{-1}$   \\ \hline
11.7            & 11.7          & 0.706837607         \\ \hline
21.73           & 10.03         & 0.705882353         \\ \hline
33.14           & 11.41         & 0.684487292         \\ \hline
42.68           & 9.54          & 0.694968553         \\ \hline
54.12           & 11.44         & 0.66520979          \\ \hline
66.06           & 11.94         & 0.654941374         \\ \hline
76.92           & 10.86         & 0.621546961         \\ \hline
87.08           & 10.16         & 0.567913386         \\ \hline
97.22           & 10.14         & 0.606508876         \\ \hline
107.36          & 10.14         & 0.371794872         \\ \hline
117.7           & 10.34         & 0.373307544         \\ \hline
131.82          & 14.12         & 0.23796034          \\ \hline
\end{tabular}
\end{table}

また、不確かの式として以下のものを利用した。

\begin{equation}
    \frac{\Delta\eta}{\eta} = \sqrt{(\frac{\Delta v_0}{v_0})^2+(\frac{2\Delta a}{a})^2+(\frac{\Delta(\cos{\theta})}{\cos{\theta}})^2+(\frac{\Delta \rho}{\rho})^2}
\end{equation}

\begin{equation}
    \frac{\Delta\gamma}{\gamma} = \sqrt{(\frac{\Delta l_0}{l_0})^2+(\frac{2\Delta a}{a})^2+(\frac{\Delta(\cos{\theta})}{\cos{\theta}})^2+(\frac{\Delta \rho}{\rho})^2}
\end{equation}

\begin{equation}
    \frac{\Delta\rho}{\rho} = \sqrt{\frac{(\Delta h_1)^2+(\Delta h_2)^2}{|h_1 - h_2|^2} + \frac{(\Delta h'_1)^2+(\Delta h'_2)^2}{|h'_1 - h'_2|^2}}
\end{equation}

\begin{equation}
    \Delta(\cos{\theta}) = (\sin{\theta)}\Delta\theta)
\end{equation}

\begin{equation}
    \Delta\frac{1}{l_0} = \frac{\Delta l_0}{l_0^2}
\end{equation}

上の式を用いて計算する際に下記の値を用いた。

\begin{table}[H]
\centering
\caption{計算をする際に用いた値}
\label{keisanti1}
\begin{tabular}{|c|c|c|c|}
\hline
                & 水       & エタノール   & 混合溶液   \\ \hline
a /cm           & \multicolumn{3}{c|}{0.017} \\ \hline
$\Delta$a /cm        & \multicolumn{3}{c|}{0.000} \\ \hline
g /m$\cdot \textrm{s}^{-2}$          & \multicolumn{3}{c|}{9.798} \\ \hline
$l_0$ /m           & 0.014   & 0.247   & 0.108  \\ \hline
$v_0$ m/s          & 0.014   & 0.120   & 0.008  \\ \hline
$\theta$   & 60.300  & 60.200  & 60.200 \\ \hline
$\Delta$ h         & \multicolumn{3}{c|}{0.005} \\ \hline
$\Delta$ $v_0$        & 0.066   & 0.070   & 0.043  \\ \hline
$\Delta$ $l_0$        & 0.709   & 0.435   & 0.631  \\ \hline
\end{tabular}
\end{table}

計算した結果は下記のようになった。

\begin{table}[H]
\centering
\caption{粘性率,表面張力}
\label{my-label}
\begin{tabular}{|l|l|l|l|}
\hline
     & 水             & エタノール         & 混合溶液          \\ \hline
粘性率 $/\textrm{Pa}\cdot \textrm{s}$ & $0.01230 \pm 0.004$ & $0.001139 \pm 0.0006$ & $0.01933 \pm 0.005$ \\ \hline
表面張力 /$\textrm{dyn}\cdot \textrm{cm}^{-1}$ & $5.8076 \pm 0.5$ & $79.9115 \pm 0.04$ & $38.561 \pm 0.8$ \\ \hline
\end{tabular}
\end{table}


\section{考察}

\subsection{データの正確性について}
エタノールの密度は純正化学株式会社のエタノールの安全データシートのデータによると0.791 $\textrm{g}/\textrm{cm}^3$なので実験によって得られた測定値との相対誤差を計算すると、$\frac{0.791 - 0.787}{0.791} = 0.00417$となるので密度は非常に近い値をだしていると考察する。また、同様に粘成立については純正化学社のエタノールの安全データシートによると、粘度は1.074 mPasなのでこの値とも実験結果の値の相対誤差を取ると次のような値となった。計算式は$\frac{1.114 - 1.074}{1.074} = 0.3724$である。この相対誤差は少し大きいと考察する。

\subsection{結果から}
この結果から、水とエタノールの場合は、それらを混合したものの粘性率はそれらの粘性率の間のものとなるのではなく、それらの値よりも大きいものになるということがわかった。この結果から、混合したものの粘性率の値は混ぜる前のそれぞれの粘性率の値の間を取るわけでは無いと考察する。




\begin{thebibliography}{99}
    \bibitem{UEC} 共通教育部自然科学部会(物理)、『基礎科学実験A(物理学実験)平成29年(2017年)版』
    \bibitem{MSDS} 純正化学株式会社 エタノール(99.5) 安全データシート
\end{thebibliography}


\newpage

\section{再レポート部分}

\subsection{結果}

グラフに不備があると感じたので再びグラフを描いた。グラフから得られる値は下記のようになった。

\begin{table}[H]
\centering
\caption{グラフより得られた値}
\label{my-label}
\begin{tabular}{|l|l|l|l|}
\hline
   & 水        & エタノール  & 混合溶液     \\ \hline
$l_0$ /m & 0.1647   & 0.0754 & 0.11454  \\ \hline
$v_0$ /m$\cdot\textrm{s}^{-1}$& 0.017146 & 0.0117 & 0.007743 \\ \hline
\end{tabular}
\end{table}

新しく得られた以上の結果と\ref{keisanti1}表の値を用いて次のように計算した。\\
粘性率$\eta = \frac{(0.01685\times 10^{-2})^2}{8\times 0.01714}\times 1000\times 9.798\times 0.4955 = 0.001004$\\
表面張力$\gamma = \frac{1}{2}\times 0.0001685\times 0.1647\times 1000\times 9.798\times 0.4955 = 0.06736$\\
以上の2式は水での値を用いたものだが、エタノール,混合溶液の計算も同様に行い、得られた値は下表のようになった。

\begin{table}[H]
\centering
\caption{計算結果の表}
\label{kekka1}
\begin{tabular}{|l|l|l|l|}
\hline
      & 水              & エタノール          & 混合溶液           \\ \hline
表面張力$\gamma$ /$\textrm{N}\cdot \textrm{m}^{-1}$ & 0.06736    & 0.02426    & 0.04102   \\ \hline
粘性率$\eta$ /$\textrm{Pa}\cdot\textrm{s}$  & 0.001004 & 0.001158 & 0.001948 \\ \hline
\end{tabular}
\end{table}

\subsection{考察}

文献値との比較を行う。実験結果と文献値を下表にまとめた。また、不確かさの値は単位などを再考した上で計算したものを載せている。

\begin{table}[H]
\centering
\caption{実験結果と文献値}
\label{my-label}
\begin{tabular}{|l|l|l|l|l|}
\hline
      & 粘性率$\eta$/$\textrm{Pa}\cdot\textrm{s}$       & 粘性率$\eta$(文献値)/$\textrm{Pa}\cdot\textrm{s}$  & 表面張力$\gamma$/$\textrm{N}\cdot \textrm{m}^{-1}$   & 表面張力$\gamma$(文献値)/$\textrm{N}\cdot \textrm{m}^{-1}$ \\ \hline
水        & $0.0010 \pm 0.0004$   & 0.0010016   & $0.0674 \pm 0.0005$   & 0.072736   \\ \hline
エタノール  & $0.00115 \pm 0.00006$ & 0.001084     & $0.02426 \pm 0.00004$ & 0.02227    \\ \hline
混合溶液   & $0.00195 \pm 0.0005$   &             & $0.0410 \pm 0.0008$   &            \\ \hline
\end{tabular}
\end{table}

上表より水の粘性率は不確かさを含めた範囲の中に文献値があることがわかる。従って、水の粘性率については正しく求められた考察できるのでこの部分については実験は成功したと考察することができる。一方、その他の値に関しては不確かさを含めた範囲の中に文献値が収まることがなかった。webクラスでは「粘性率は水やアルコールがクラスタ構造を持つため40\%〜50\%の近辺で極大」とある。実際に実験結果では粘性率は混合溶液の場合において水の場合とエタノールの場合とを比較すると、最大になっている。従ってこの実験において得られた粘性率は大きく外れてはいないと考察する。同様にwebクラスでは「表面張力は内部構造よりも界面の状態によるので単調減少する」とあり、今回の実験結果においてもそのようになっている。よって今回の実験によって得られた表面張力は大きく外れてはいないと考察することができる。


\begin{thebibliography}{99}
    \bibitem{nenpyo} 国立天文台 理科年表 平成30年
\end{thebibliography}
参考にしたwebサイト
    webclass:\url{https://webclass.cdel.uec.ac.jp/webclass/login.php?acs_=41394799}
\end{document}